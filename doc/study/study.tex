\documentclass[a4]{article}

\usepackage{booktabs}
\usepackage{siunitx}

\author{Javed Mohamed}
\title{BVH Study}

\begin{document}
\maketitle

\begin{table*}[h]
  \centering
  \begin{tabular}[h]{rrr}
    \toprule
    & \multicolumn{2}{c}{Time (\si{\second})}\\
    \cmidrule(l){2-3}
    Spheres & Un-Accelerated & Accelerated \\
    \midrule
    100 & 0.33 & 0.11 \\
    1000 & 2.54 & 0.15 \\
    10000 & 32.83 & 0.50 \\
    \bottomrule
  \end{tabular}
  \caption[Ray-tracer runtimes]{Runtimes for accelerated and non-accelerated
    ray-tracers.}
  \label{tab:runtimes}
\end{table*}

The bounding volume hierarchy (BVH) used in this project uses axis-aligned
bounding boxes (AABB). The BVH is conceptually created by first wrapping each
geometric primitive (e.g. spheres and triangles) in an AABB. Then the list of
elements is sorted based on the bounding box's position is space by dimension
(i.e. sorting by the $x$-, $y$-, or $z$-axis). Then the list is split in half,
forming the left and right halves. This process is recursively done until the
leaf nodes are just geometric primitives. This creates a tree that can be hit.
The idea is that we should be able to quickly discard many objects using a cheap
hit function, then use a more expensive one once we have limited out options to
a few primitives. This means we approach $\mathcal{O}(\log n)$ time complexity.
Spheres were uniformly randomly generated within a cubic volume in the center of
the image. The image is rendered using a perspective camera.

\end{document}